\chapter{Conclusion}
\label{c:conclusion}
This research concentrated on streaming of high-resolution progressive meshes over network. 
The objective is to improve the quality of rendered image in the receiver side
as quick as possible, or in other words, to improve the quality curve in the receiver
side. 

This thesis contributes to this objective in four aspects.
First, we try to quantize and then minimize the negative effect of dependency
among the vertex splits when packet losses happen during transmission.
Second, we proposed a more scalable way to implement view-dependent streaming
system, which can significantly improve the rendering quality curve by considering 
users' viewpoints. 
%With our receiver-driven protocol, view-dependent steaming system
%is easier to be adapted to a large number of users.
Third, we try to understand the user interaction on progressively transmitted 
high-resolution meshes, and then we can use prediction and caching to further 
improve the quality curve. Moreover, arbitrary large number of synthetic traces 
can be generated to further tune the system to achieve better performance.
Finally, with the help of the receiver-driven protocol we proposed, we developed 
a prototype of P2P view-dependent progressive mesh streaming system. With P2P
technique, a receiver can request data from many peers to improve the quality curve 
quickly.

In this chapter, we briefly review the main contributions of this thesis and point
out the possible future work.

\section{An Analytical Model to Quantize the Effect of Dependency and Packet Loss}
First, the effect of dependency over lossy network was quantitatively analyzed. 
To enable quantitative analysis, an analytical model was developed to compute 
the probability distribution of the decoding time of each vertex split. 
Many experiments were carried out to verify the accuracy of the analytical model, 
and different network traces and progressive meshes were used to avoid bias. 
In these experiments, the predicted decoding time of vertex splits obtained 
with the analytical model was reasonably close to the measured one. 
Therefore, with the help of the analytical model, 
the quality of the received mesh could be predicted efficiently before transmission.
Consequently, adaptive transmitting vertex splits according to the network condition
becomes possible. Besides streaming of progressive meshes, 
streaming of other partially ordered data could also be modeled
with the analytical model in this thesis. 
For example, this model was successfully applied in streaming of 3D plants.

The analytical model in this thesis assumes that the packet losses 
are independent from each other, 
so a network trace with many burst packet losses may cause considerable error in the prediction. 
Gilbert model of packet loss may be used to obtain results that are more accurate, 
but using Gilbert model may significantly increase the complexity of the analytical model. 
Moreover, burst packet losses happen less commonly in the Internet
after widespread application of RED in routers. 
Therefore, the precision of the analytical model
without considering burst packet loss may be acceptable in most cases now.

It is observed in the experiments that the effect of dependency
is only significant in the short period at the beginning of transmission, 
and the length of this period increases
when the round trip time or packet loss rate increases. 
This observation is consistent with the result of the analytical model. 
Since each lost packet will be retransmitted as soon as the packet loss is detected, 
the effect of a packet loss is restricted to a certain period. 
This effect decreases with time because of two reasons. 
First, the contribution of the vertex splits sent decreases; 
second, the quality of the reconstructed mesh increases. 
Consequently, the relative value of the effect, 
represented as the ratio of the contribution of vertex
splits to the quality of the reconstructed mesh, decreases.
As a result, considering the dependency among the vertex splits 
is only needed in applications requiring high interactivity
because these applications require low initial latency. 
For these applications, the greedy method proposed in this thesis
could generate an adaptive sending strategy to reduce the initial latency. 

\section{Receiver-Driven View-Dependent Mesh Streaming Protocol}
View-dependent streaming is needed in streaming of high-resolution meshes
to improve the quality quickly, but current implementations cannot be easily scaled
to large number of receivers as the sender is not stateless. 
To address this weakness, a receiver-driven view-dependent streaming system,
in which the sender is stateless, was proposed in this thesis. 
Moving the visibility decision to the receiver reduces the overhead of the server. 
Moreover, a significant advantage of receiver-driven approach is that
now both the caching proxy and P2P techniques can be applied to increase the scalability of the system
since the sender is stateless. 

One main challenge of receiver-driven protocol is enable the receiver
to determine the visibility and visual contribution of the vertices.
We proposed to use the screen area as the estimation of visual contribution and 
experimental results show that this method works well. 
We also proposed a method
to detect the silhouette and high priority could be assigned to silhouette when the
profile of the mesh is important. 
There is, however, no satisfying objective quality metric to measure the quality
of rendered image, which balance well between avoiding large block error and 
keeping correct silhouette. To provide a better quality metric could be an interesting 
future work. 

Further research could investigate into better way to estimate the visual importance.
For example, we could consider the roughness of the mesh, the transient level of the mesh,
or the semantic meaning of different parts of the mesh, so that we could refine 
the part which has most importance.

The visibility decision made by the receiver may not be accurate 
since the receiver can only estimate the visibility based on
the partially received mesh. 
Visible vertices may be not displayed if their ancestor vertices are not visible. 
According to the current results, however, this kind of error is negligible in most cases. 
Moreover, we found that force splitting the neighbors of vertices in silhouette can significantly reduce this error.
For applications that tolerate no error, 
further research could be done to remove this kind of error. 

\section{P2P View-Dependent Progressive Mesh Streaming}
To increase the scalability of the view-dependent mesh-streaming systems, 
a P2P mesh streaming system was proposed in this thesis based on the receiver-driven approach. 
According to the experimental results, the server overhead can be reduced by 90\%
and the response time is acceptable. %still close to the low bound. 
The control overhead introduced in this system is about 10\%, a reasonable value. 
%The response time is low because most requests in this system are successful in the first try, 
%and the message overhead is reasonable because the hierarchical group system proposed in this thesis effectively 
%reduces the message overhead. 
The prototype developed in this thesis shows that P2P mesh streaming system is feasible
and it can significantly reduce the server cost when there are many clients.

The control overhead at 10\% is acceptable in this system since the majority of
control messages are transmitted among the peers and the server overhead is kept low. 
Nonetheless, the control overhead could be further reduced in the future. 

The response time less than one second is acceptable in many application without 
strict real time requirement. For those application where do require low latency, such
as 3D games, pre-fetching can help to reduce the response time. It could be an interesting 
research topic to improve the P2P system with the insight from our user study.

In a view-dependent progressive mesh streaming system, the interaction of users directly
affect which data is streamed and in which order they are streamed.
We have done some preliminary study on how users interact 
with a progressively streamed mesh.
%With the analysis of user records, we found that user actions are 
%somewhat predictable since the user tends to repeat the previous 
%action. 
%Therefore, pre-fetching becomes possible and it can be used both in streaming and rendering to
%reduce response time.
%Furthermore, the analysis of user traces reveals that locality exists in both data and viewpoint access. 
%We can exploit this locality and use caching technique in progressive mesh streaming systems and rending systems.

%After studying the user records, 
We proposed a simple model 
to generate synthetic traces based on real traces we collected. 
These synthetic traces have some similar properties with the real traces and 
could be used in simulations to measure system performance.
Due to the limited number of user traces collected, we have to 
sacrifice the accuracy in this simple model. In the future, if we could develop a real 
online shop or online museum, we could collect a large number of real traces.
With more real traces, we could develop a more accurate model.

\section{Future Work in Other Topics}
In addition to the four aspects we investigated into, more interesting topics exist to be examined.
For example, we found that in some cases the bottleneck can be the rendering of the meshes in the local 
terminal, especially when the GPU is also used to decide the visibility of vertices. 
Improving the rendering efficiency of the progressive mesh during the streaming 
could be an interesting research problem.  
During streaming, a progressive mesh with only partial modification is frequently re-rendered, 
so it is possible to exploit the similarity between two consecutive meshes to improve the rendering speed.

Moreover, remote rendering is an effective way to enable devices without enough resources to 
join the mesh streaming system. An extended P2P systems, where some peers not only provide the raw vertex splits
but also provide rendered images, could be developed to further enable the 3D streaming applications on resource
constraint devices.
%\end{comment}
