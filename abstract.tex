\begin{abstract}
    %\begin{comment}
    High-resolution 3D meshes are increasingly available in networked
    applications, such as digital museum, online game, and virtual reality.
    The amount of data constituting a high-resolution 3D mesh can be
    huge, leading to long downloading time. 
    To reduce the waiting time of users, 
    a common technique for remote viewing is progressive streaming,
    which allows a low-resolution version of the mesh to be transmitted
    and rendered with low latency. The quality of the transmitted 
    mesh is incrementally improved by continuously transmitting the refinement information.

    Progressive mesh is commonly used to support progressive
    streaming. A progressive mesh comprises a \emph{base mesh} and a series
    of refinements. The base mesh is obtained by simplifying the original mesh
    with a series of \emph{edge collapses}.
    The inverses of these edge collapses, named \emph{vertex splits}, 
    can reconstruct the original mesh from the base mesh.
    Therefore, progressive streaming can be implemented by sending the vertex
    splits as refinements after sending the base mesh.

    %Although streaming of videos is extensively studied, 
    Streaming of high-resolution progressive meshes is considerably different
    to video streaming, which has been extensively studied.
    Frames of a video are usually sent following the time order.
    Vertex splits of a progressive mesh, however, can be sent in various orders.
    New research problems arise due to the flexibility in sending order of 
    vertex splits. In this thesis, three such problems are addressed. 
    %After decoding, each frame contributes the same to a video, but vertex splits
    %in a progressive mesh vary considerably in their contribution to the quality
    %of the reconstructed mesh. 
       
    First, 
    %the effect of dependency needs to be considered in choosing
    %the sending order of vertex splits.
    the progressive coding of meshes introduces dependencies among 
    the vertex splits, and the descendants cannot be decoded
    before their ancestors are all decoded. Therefore, 
    when a progressive mesh is transmitted over a lossy network,
    a packet loss will delay the decoding of any following
    vertex split that depends on a vertex split in this lost packet. 
    %These successfully received vertex splits cannot be 
    %decoded until the lost packet is successfully retransmitted. 
    Hence, the effect of dependency needs to
    be considered in choosing the sending order of vertex splits. 
    %To find the effect of dependency
    %is non-trivial as the packet loss happens randomly.
    In this thesis, an analytical model is proposed to 
    %quantify the effect of dependency on the quality curve.
    quantitatively analyze the effects of dependency
    by modeling the distribution of decoding time of each vertex split
    as a function of mesh properties and network parameters.
    %The quality curve is determined by the decoding time of the vertex
    %splits and their contributions to the overall mesh quality.    
    %The decoding time of a vertex split depends on its receiving time and the 
    %receiving time of its all ancestors. The receiving time of a vertex split
    %in turn depends on its sending time and how many times it is 
    %transmitted. 
    %Hence, the distribution of decoding time of each vertex split
    %is modeled as a function of the mesh property and the network condition.
    %From this expression, %both the expected value and 
    %the distribution of the decoding time of each
    %vertex split can be efficiently computed before transmission. 
    Consequently, different sending orders can be  
    efficiently evaluated without simulations, 
    and this model can help in developing a sending
    strategy to improve the quality curve 
    during transmission. % to improve the quality.
    
    Furthermore, %closed-form expressions are derived in two extreme cases,
    this model is applied to study two extreme cases of dependency in 
    progressive meshes. %and
    %providing insights to the importance of dependencies on the
    %decoded mesh quality. 
    The main insight we found is that if each lost packet
    is immediately retransmitted upon the packet loss is detected, 
    the effect of dependencies on decoded mesh quality diminishes with time.
    %the dependencies only matters in the first several round trip times. 
    Therefore, the effect of dependency is only significant in the applications
    requiring high interactivity. 

    The accuracy of the analytical model proposed in this thesis is validated
    under a variety of network conditions, including bursty losses, fluctuating RTT,
    and varying sending rate. The values predicted from our model match the measured
    value reasonably well in all cases except when losses are too bursty.
    
    Second, to improve the quality of rendered image in the receiver side quickly, 
    the viewpoint of the user can be considered in deciding the sending order.
    %Usually, users can only see a part of a mesh, so sending non-visible data
    %before visible data wastes bandwidth. Moreover, 
    Non-visible vertex splits do not need to be sent, and 
    the visual contributions of visible vertex splits (how much they can improve the rendered image)
    also depend on the viewpoint of the user.
    Hence, choosing the sending order according to the viewpoint of the user, so called
    \emph{view-dependent streaming}, is a natural choice.
    
    In existing solutions to view-dependent streaming,  the
    sender decides the sending order. 
    %This sender-driven protocol has several
    %drawbacks. First, it is not scalable to many receivers as the sender has to
    %determine the visibility of vertices and sort the visible vertex splits based on their
    %visual contributions for each receiver. 
    %Second, 
    The sender needs to maintain
    the rendering state of each receiver to avoid sending duplicate data. 
    Due to the state-ful design, %and high computational requirements,
    the sender-driven approach cannot be easily extended to support 
    many receivers with caching proxy and peer-to-peer (P2P) system,
    two common solutions to scalability. 

    In this thesis, a receiver-driven protocol is proposed to 
    improve the scalability. In this protocol, the receiver decides the sending order
    and explicitly requests the vertex splits, 
    while the sender simply sends the data requested.
    The sending order is computed at the receiver  by estimating the visibility
    and visual contributions of the refinements, even before receiving them,
    with the help of GPU.

    Because the visibility determination and state maintenance are all done by the receivers, 
    the sender in our receiver-driven protocol is stateless and free of complex computation.
    Furthermore, caching proxy and P2P streaming systems can be applied to improve
    the scalability without adding more servers.  

    
    Third, based on the receiver-driven protocol we proposed, 
    P2P techniques are applied to view-dependent progressive mesh streaming in this thesis. 
    In the implementation of P2P mesh streaming system,
    two issues are considered: how to partition a progressive mesh into chunks and 
    how to lookup the provider of a chunk. For the latter issue, we investigated 
    two solutions, which trade off server overhead and response time. The first
    uses a simple centralized lookup service, while the second organizes peers into
    groups according to the hierarchical structure of the progressive mesh to 
    take advantage of access pattern. 
    We have implemented a prototype and test its performance
    with synthetic traces we generated based on real traces logged from 37 users.
    %with the synthetic traces we generated. 
    %The synthetic traces are generated with the simple model we proposed based on the user study we have done. 
    Simulation results show that our proposed systems are robust under high churn rate. It reduces the server overhead
    by more than $90\%$, keeps control overhead below $10\%$, and achieves
    average response time lower than 1 second.
    
    %When a progressive mesh is streamed in a view-dependent way, which set of the data is streamed,
    %and in what order it is streamed, relate to how the user interacts with the mesh. 
    %To test the system performance of a P2P system, 
    %it is desirable that a large number of user traces,
    %recording how users interact with the mesh, 
    %are collected and replayed in the simulation to obtain the more realistic results. 
    %Collecting a large number of user traces is expensive and not always feasible.
    %Instead, we have collected some traces from 37 real users 
    %and did some preliminary analysis on them and then 
    %developed a simple model, which can be used to generate synthetic traces
    %for testing progressive mesh streaming systems.
%\end{comment}
\end{abstract}
